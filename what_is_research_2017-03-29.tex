\documentclass{beamer}

% theme
\mode<presentation>
{
\usetheme{Warsaw}

\setbeamercovered{transparent}
}

% packages
\usepackage{hyperref}
% \usepackage{multicol}

% setting
\linespread{1.2}

% title page
\title{What is Research?}
\author{Guo Zhang}
\institute{WISE, Xiamen University}
\date{This Version: \today}
\subject{Presentations}

% body
\begin{document}

\begin{frame}
\maketitle
\end{frame}

\begin{frame}[plain]
\frametitle{Contents}
% \begin{multicols}{2}
\tableofcontents %[hideallsubsections]
% \end{multicols}
\end{frame}

\section{What is Research?}
\begin{frame}
\frametitle{A Simple Example}
\begin{itemize}
   \item \url{www.tmall.com}
\end{itemize}
\end{frame}

\begin{frame}
\frametitle{What is Research?}
\begin{itemize}
	\item Define a problem
	\item Solve the problem
\end{itemize}
\end{frame}

\begin{frame}
\frametitle{Why do we need different subjects?}
\begin{itemize}
	\item A problem can be detached from different angles;
	\item A single angle cannot solve the problem perfectly.
\end{itemize}
\end{frame}

\begin{frame}
\frametitle{Why does research become an occupation?}
\begin{itemize}
	\item Problems are more complex and complicated, which makes specialization necessary;
	\item Methodologies of research have been developed rapidly in the recent decades.
\end{itemize}
\end{frame}

\begin{frame}
\frametitle{Why is research valuable?}
\begin{itemize}
	\item Public goods
	\item Increasing return to scale.
\end{itemize}
\end{frame}

\section{How to Organize a Research?}
\begin{frame}
\frametitle{A General Process}
\begin{itemize}
	\item Plan
	\item Implement
	\item Present
\end{itemize}
\end{frame}

\begin{frame}
\frametitle{How to Plan?}
\begin{itemize}
	\item Choose a topic
	\item Read literatures
	\item Make feasibility analysis
	\item Design methods
	\item Write a proposal
\end{itemize}
\end{frame}

\begin{frame}[allowframebreaks]
\frametitle{How to Implement?}
\begin{itemize}
	\item Theory
	\begin{itemize}
		\item Construct assumptions
		\item Deduce key conclusions
		\item Discuss properties
		\item Extend models	
	\end{itemize}
	\item Experiment 
	\begin{itemize}
		\item Design experiments
		\item Do experiements
		\item Report results
	\end{itemize}
	\framebreak
	\item Data
	\begin{itemize}
		\item Collect data
		\item Clean data 
		\item Describe data 
		\item Fit models
		\item Report results
	\end{itemize}
\end{itemize}
\end{frame}

\begin{frame}
\frametitle{How to Present?}
\begin{itemize}
	\item Write papers/reports
	\item Present on meetings/competitions
	\item Publish papers/reports
\end{itemize}
\end{frame}


\section{My Research Journal}
%\subsection{Timeline}

\begin{frame}
\frametitle{First Year}
\begin{itemize}
	\item 2015-04, New Angle, Member
	\item 2015-05, Qincheshulu Social Practice Team, Member
	\item 2015-05, China Internet Project at UCB, Research Assisitant
	\item 2015-06, WISE Book Club, Organizer
\end{itemize}
\end{frame}

\begin{frame}
\frametitle{Second Year}
\begin{itemize}
	\item 2015-08, 985 Staffs Database Project, Member
	\item 2015-11, 2015 China’s Social-economic Development Analysis Tournament by Harvard, Research Assistant
	\item 2016-02, China's Prices Project, Founder and Teamer Leader
	\item 2016-03, Prof. Zhigang Qin, Second Author and Technical Leader
	\item 2016-04, Top Plan in Economics, Member
	\item 2016-05, Zuqizhisheng Social Practice Team, Adviser 
	\item 2016-06, WISER Club, Co-president
\end{itemize}
\end{frame}

\begin{frame}
\frametitle{Third Year}
\begin{itemize}
	\item 2017-02, Python China, Co-partner
	\item 2017-03, Prof. Thomas Sargent, Research Assistant
	\item 2017-03, QuantEcon, Contributor 
\end{itemize}
\end{frame}


\begin{frame}[plain]
\frametitle{Contents}
% \begin{multicols}{2}
\tableofcontents %[hideallsubsections]
% \end{multicols}
\end{frame}

\end{document}